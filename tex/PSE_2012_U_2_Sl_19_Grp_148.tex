\documentclass{scrartcl}
\usepackage{fontspec}

\usepackage{bashful}
\usepackage{xcolor}
\usepackage{listings}

\usepackage{fancyhdr}
\usepackage{minted}

\title{PSE_2012_U_2_Sl_19_Grp_148}
\author{Clemens Mayer}
\date{\today}

\pagestyle{fancy}
\lhead{Peter Faber (2836641)}
\chead{Clemens Mayer (2815170)} 
\rhead{Han Ngoc Phan (2821355)}

\lstdefinestyle{BashOutputStyle}{
  basicstyle=\small\ttfamily,
  numbers=none,
  linewidth=\linewidth,
  xleftmargin=0.1\linewidth,
}

\begin{document}
%\begingroup
\catcode`\_=12\relax
/Users/clemens/Documents/uni/PSE/gruppe_148/Hausaufgaben-Blatt-2/ISTEMediaPlayer.java
\inputminted{java}{/Users/clemens/Documents/uni/PSE/gruppe_148/Hausaufgaben-Blatt-2/ISTEMediaPlayer.java}
/Users/clemens/Documents/uni/PSE/gruppe_148/Hausaufgaben-Blatt-2/PSE_2012_U_2_Sl_19_Grp_148.java
\inputminted{java}{/Users/clemens/Documents/uni/PSE/gruppe_148/Hausaufgaben-Blatt-2/PSE_2012_U_2_Sl_19_Grp_148.java}
\endgroup


\section*{Aufgabe 4}
%\bash[script,stderrFile=/Users/clemens/Documents/uni/PSE/gruppe_148/Hausaufgaben-Blatt-2/tex/PSE_2012_U_2_Sl_19_Grp_148.stderr,scriptFile=/Users/clemens/Documents/uni/PSE/gruppe_148/Hausaufgaben-Blatt-2/tex/script.sh,stdoutFile=/Users/clemens/Documents/uni/PSE/gruppe_148/Hausaufgaben-Blatt-2/tex/PSE_2012_U_2_Sl_19_Grp_148.stdout,exitCodeFile=/Users/clemens/Documents/uni/PSE/gruppe_148/Hausaufgaben-Blatt-2/tex/PSE_2012_U_2_Sl_19_Grp_148.exitCodes]
%java ISTEMediaPlayer
%\END
\lstinputlisting[style=BashOutputStyle]{/Users/clemens/Documents/uni/PSE/gruppe_148/Hausaufgaben-Blatt-2/tex/PSE_2012_U_2_Sl_19_Grp_148.stdoutt}
\section*{Aufgabe 5}
\begin{tabular}{|p{0.33\textwidth}|}
\hline
	class BugsBunnyMusik\\
\hline
	- Title: String\\
	- Artist: String\\
	- Album: String\\
	- duration: int\\
\hline
	+ play(): void\\
	+ play(seconds: int): void\\
	+ getDiscription(): void\\
	+ getDuration(): void\\
	+ loop(): void\\
	+ getTitle(): String\\
	+ getArtist(): String\\
	+ getAlbum(): String\\
	+ setTitle(): void\\
	+ setArtist(): void\\
	+ setAlbums(): void\\
\hline
\end{tabular}
\begin{tabular}{|p{0.33\textwidth}|}
\hline
	class MaschenDrahtSong\\
\hline
	- Title: String\\
	- Artist: String\\
	- Album: String\\
	- duration: int\\
\hline
	+ play(): void\\
	+ play(seconds: int): void\\
	+ getDiscription(): void\\
	+ getDuration(): void\\
	+ loop(): void\\
	+ getTitle(): String\\
	+ getArtist(): String\\
	+ getAlbum(): String\\
	+ setTitle(): void\\
	+ setArtist(): void\\
	+ setAlbums(): void\\
\hline
\end{tabular}
\begin{tabular}{|p{0.33\textwidth}|}
\hline
	class TourSong\\
\hline
	- Title: String\\
	- Artist: String\\
	- Album: String\\
	- duration: int\\
\hline
	+ play(): void\\
	+ play(seconds: int): void\\
	+ getDiscription(): void\\
	+ getDuration(): void\\
	+ loop(): void\\
	+ getTitle(): String\\
	+ getArtist(): String\\
	+ getAlbum(): String\\
	+ setTitle(): void\\
	+ setArtist(): void\\
	+ setAlbums(): void\\
\hline
\end{tabular}



\end{document}
